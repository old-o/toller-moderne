% Examensarbeit: Dramenanalyse "Die Wandlung"

\HeadingOne{Frühe Generalisierung des Gemeinschaftsideals:\\ 
  \Cite{Die Wandlung}}{\Cite{Die Wandlung}}

Das Stück ist wohl das Werk Tollers, das von der Forschung am meisten
interpretiert worden ist und entscheidend zur Einstufung des Autors als
Vertreter des \Cite{messianischen Expressionismus} beigetragen
hat.\Footnote{So \BibRef{vietta.94}{200}: \Cite{Toller ist der wohl wichtigste
    Dramatiker eines in politischen Aktivismus mündenden expressionistischen
    Verkündungsdramas.} \Abr{Vgl} auch \BibRef{reimer.00}{46}.}  Es ist nicht selten
als Vorlage für eine vergleichende Bewertung seinerer späteren Dramen genommen
worden, in denen dann das schrittweise Abweichen Tollers vom ursprünglichen
\Cite{\Quote{O Mensch}-Pathos} konstatiert wurde.

In der vorliegenden Arbeit wird die \Title{Wandlung} relativ ausführlich
untersucht, weil darin das Grundmotiv der \Quote{Einheit und Ganzheit in
  (revolutionärer) Gemeinschaft}, das für Tollers spezifische Form der
Modernebewältigung charakteristisch ist, in Reinform zur Darstellung
kommt.

\HeadingTwo{Inhaltliche Zusammenfassung}

Der Titel \Title{Die Wandlung} ist programmatisch für Tollers dramatischen
Erstling: Mit dem Protagonisten des Stücks durchläuft ein Mensch sozusagen
exemplarisch für die Menschheit eine innere Wandlung hin zum wahren Menschsein
und trägt die gewonnene Erleuchtung dann an die Masse der Menschen, das
\Cite{Volk}, weiter.

Die Suche der Hauptfigur Friedrich nach Sinn und Gemeinschaft jenseits seiner
als beengt empfundenen sozialen Herkunft ist der Ausgangspunkt der
Handlung. Die Ausrichtung auf vaterländische Kriegsbegeisterung in der
Hoffnung auf Integration in die nationale Gemeinschaft bildet den zunächst
unternommenen Lösungsversuch. Friedrichs freiwillige Kriegsbeteiligung und
sein zu Heldenehren führender soldatischer Wagemut sind die handlungslogischen
Schritte, die den ersten Bildern und Stationen des Dramas ihre Richtung
geben. Das von Friedrich erlebte Kriegsleiden wird kontrastierend dargestellt,
die Gnadenlosigkeit der Kriegslogik und die nur fiktive Gemeinschaft des
\Cite{Vaterlandes} lassen den Protagonisten zweifeln und markieren die
Entwicklungsrichtung zunehmend als Irrweg. Die Ausbeutung von
Arbeitern und Soldaten, das Leiden der Kriegsopfer sowie die ungleiche
Verteilung von Macht und Reichtum diskreditieren die vermeintlich
identitätsstiftende Idee der Nationalgemeinschaft noch zusätzlich.

In einem qualvollen Reinigungsprozess, der durch Impulse inspirierter
Mittlerfiguren und explizite Bezugnahme auf Passionsmotive den Rang einer
höheren Notwendigkeit erhält, wird Friedrich schließlich auf den rechten
Weg \Cite{zu den Menschen} geschickt: Die Liebe zu den Menschen, die
Wiedergeburt und Selbsterkenntnis der Menschheit werden so zur Mission des
fündig gewordenen Sinnsuchers. Die zentrale Szene seiner inneren Wandlung
vollzieht sich als symbolüberladene \Cite{Selbstkreuzigung} und \Cite{Geburt}
an einem allegorisch verfremdeten Ort, der Gefängnis und \Cite{Fabrik}
zugleich ist \SourceRef{II}{43-46}.\Footnote{Die Angaben in Klammern stehen
  abkürzend für \Cite{Band II, Seite 43 bis 46} und beziehen sich auf die
  \Title{Gesammelten Werke}. Soweit nicht anders vermerkt, beziehen sich im
  Folgenden alle Seitenangaben für Primärtexte auf diese Quelle.}
Der danach zum zielstrebigen Akteur gewordene \Cite{Wanderer} \SourceRef{II}{46}
bringt seine derart erlangte Läuterungsstufe in Form von Volksreden in die
öffentliche Wahrnehmung. Die Verkündung seines \Quote{Wissens um die
  Menschen}\Footnote{\Abr{Vgl} \SourceRef{II}{40}: \Cite{Meine Augen schauen
    den Weg. [..] Allein, und doch mit allen, / Wissend um den Menschen.},
  sowie \SourceRef{II}{58}: \Cite{Keinen von euch kenne ich und doch weiß ich
    um euch alle.}}
und ihre Leiden und Schwächen mündet am Dramenschluss nahtlos in die
begeistert aufgenommene Ausrufung einer friedlichen Revolution, die ihre
euphorische Grundlage in der schlichten und gerade darin für die Beteiligten
so erschütternden Erkenntnis ihres eigentlichen und wahren \Cite{Menschseins}
findet \SourceRef{II}{60}.

\HeadingTwo{Bemerkungen zu Wirkungsabsicht und Textgestalt}

Ernst Toller selbst hat in einem rückblickenden Kommentar betont, sein erstes
Drama sei ihm in der Zeit seiner Entstehung \Cite{Flugblatt} gewesen, aus dem
er vorlas, um \Cite{Dumpfe aufzurütteln, Widerstrebende zum Marschieren zu
  bringen, Tastenden den Weg zu zeigen}. In Richtung auf eine politische
Dramatik, die \Cite{aus der Unbedingtheit revolutionären Müssens}, in einer
\Cite{Synthese aus seelischem Trieb und Zwang der Vernunft} zu entstehen habe,
sei das Stück \Cite{vielleicht ein brüchiger Schritt} gewesen. Das
entscheidende Merkmal politischer Dramatik war für Toller die tatsächliche
\Cite{umpflügende und aufbauende} Wirkung der Werke. Um den \Cite{geistigen
  Inhalt menschlichen Gemeinschaftslebens} zu erneuern, brauche es den
\Cite{politischen Dichter}, der sich \Cite{verantwortlich fühlt für jeden
  Bruder menschheitlicher Gemeinschaft} und damit \Cite{stets irgendwie
  religiöser Dichter} sei.\Footnote{Ernst Toller: Bemerkungen zu meinem Drama
  \Cite{Die Wandlung}. In: Der Freihafen. Blätter der Hamburger Kammerspiele
  II. 1919. \Page{145}f. Zitiert nach \LongSourceRef{II}{360-61}.}

Die Verkündigungssabsicht des Dramas wird durch den Titel \Cite{Die
  Wandlung}, den Untertitel \Cite{Das Ringen eines Menschen}, das Leitwort
\Cite{Ihr seid der Weg} und eine vorangeschickte lyrische
\Cite{Aufrüttelung} unverkennbar markiert. Die Adressaten des Werkes sind
offensichtlich gedacht als eine zu mobilisierende Menschenmasse, die sich
selbst als \Cite{Weg} begreifen und diesen Weg gleichsam per Selbstfindung
beschreiten soll. Im Protagonisten und seinem \Cite{Ringen} hat die
explizite Verkündigungsintention des Stücks zugleich ihren exemplarischen
Identifikationspunkt und ihre seherische Leitfigur: \Cite{einen Bruder} mit
\Cite{dem großen Wissen} und dem \Cite{großem Willen}, wie es die Verse der
\Cite{Aufrüttelung} umreißen. Mit der Regieanweisung, die Handlung spiele
\Cite{in Europa vor Anbruch der Wiedergeburt} wird das Aufbruchs\-pathos und
die messianische Bedeutung des Wandlungsmotivs noch zusätzlich
unterstrichen. Die Entstehungszeit 1917/1918 lässt erkennen, wie unmittelbar
der Autor seine Kriegs- und Revolutionserlebnisse in eine politisch-ethische
\Quote{Sendungsabsicht} transformiert hat. Durch religiöse
Überhöhung versucht er seinen politischen Ideen normative Verbindlichkeit zu
geben.

Das Drama ist eine Abfolge von 13 \Cite{Bildern}, deren verbindendes Element die
Entwicklung der Hauptfigur Friedrich ist.\Footnote{\BibRef{schrei.97}{63}:
  \Cite{Mit der Übernahme des Stationendramas wird die spezifische Erlebnisform
    der Moderne, in welcher die äußere Wirklichkeit nur in Ausschnitten und in
    Bezug auf die Hauptfigur sichtbar und bedeutsam wird, diese aber seltsam
    entindividualisiert erscheint, formal umgesetzt.}}  
Die gesamte Handlung ist auf den Protagonisten zentriert und zeigt seinen Weg
in symmetrischer Abfolge:
Sein Irrweg währt bis zur Wandlungsszene in der Dramenmitte und schlägt dann um
in einen zielstrebigen Heilsweg.\Footnote{Denkler spricht von
  einem \Cite{Protagonistendrama mit Passionsstruktur}. Siehe
  \BibRef{denkle.81}{122}.}  Die Bilder sind aufgeteilt in Szenen auf Vorder-
und Hinterbühne, wobei letztere nach Tollers Regieanweisungen \Cite{in
  innerlicher Traumferne gespielt zu denken} sind und als allegorische, durch
Verfremdung verallgemeinerte, die Haupthandlung derart kommentierende
Einblendungen verstanden werden können.  Das gesamte Drama ist geprägt von
einem pathetisch-expressionistischen Sprachduktus.\Footnote{\Abr{Vgl} dazu
  insbesondere \BibRef{reimer.00}{47-48}.}

Es gibt keinen tragischen Konfliktaufbau im klassischen Sinn und die Instanz
des Gegenspielers fehlt völlig. Stattdessen ziehen sich zwei gegensätzliche
Prinzipien durch das Drama: Dem an mehreren Stellen als revolutionäre Kraft
erwähnten \Cite{Geist}\Footnote{Der Begriff ist bei Toller im Sinne der
  Landauerschen Revolutionskonzeption zu verstehen, die ausführlicher auf Seite
\pageref{LANDAUER} dargestellt wird.} steht der in den
Personalangaben genannte \Cite{Tod als Feind des Geistes} entgegen, der in
Gestalt verschiedener totenköpfig verfremdeter Figuren auftritt und außerdem
im allegorischen Vorspiel \Cite{Die Totenkaserne} \SourceRef{II}{13f.}
eine vorausgreifende
Kommentierung erfährt.\Footnote{Siehe dazu die eingehende Analyse dieses
  Prinzips ab Seite \pageref{TOD}.}

Dem dramatischen Text vorangestellt ist die erwähnte lyrische
\Cite{Aufrüttelung} \SourceRef{II}{7}, die in schwülstig-pathetischen Versen
zur visionären Perspektivierung des Stückes beiträgt:

\begin{BlockQuote}
  [Wir] hörten neben uns den Menschen schreien! [..] Europa troff, entblößt von
  Sudel [..]. Ein Bruder, der den großen Willen in sich trug, [..] Der ballte
  lodernd harten Ruf: Den Weg! [..] Du Dichter weise.
  \SourceRef{II}{9}\Footnote{Das Zitat
    nimmt einige Auslassungen vor, um den Aspekt des \emph{Künstlers als
      Wegweiser} prägnant heraus zu stellen.}
\end{BlockQuote}
In lyrisch verbrämter Form wird hier die Leitgestalt des intellektuellen
Künstlers vorgezeichnet, der den Weg aus dem großen Grauen weisen wird. Die
Autornorm, die mit dem Stück propagiert werden soll, ist so in expliziter
Weise dem als ebensolcher intellektueller Künstler dargestellten Protagonisten
beigegeben.\Footnote{Schon hierin zeichnet sich Tollers Orientierung an Gustav
  Landauers Lehre von den \Cite{Geistigen} ab, die die Kraft zur
  Menschheitsrevolution in sich tragen. Landauer in seinem \Quote{Aufruf zum
    Sozialismus}:  \Cite{Der Geist zieht sich in die
    Einzelnen zurück [..] , die sich in all ihrer Mächtigkeit verzehren, die ohne
    Volk sind: vereinsamte Denker, Dichter und Künstler [..]}. Zitiert nach
  \BibRef{ruehle.73}{905}. }

\HeadingTwo{Das in seiner Integrität bedrohte Subjekt}

In der Ausgangssituation der \Title{Wandlung} artikuliert sich der
Protagonist Friedrich als verzweifelte, von innerlicher \Cite{Zerrissenheit} 
geplagte Figur \SourceRef{II}{17}. Seine Unzufriedenheit tritt ausbruchsartig
zum Vorschein, er
überzieht seine Mutter mit Vorwürfen und bringt die Krisis seiner Seelenlage
geballt zum Ausdruck. Er empfindet sich als ein \Cite{Ausgestoßner}, der in
seiner Herkunftswelt keine Zugehörigkeit verspürt. Die Szenerie spielt am
Weihnachtsabend, der bei \Cite{denen drüben} gefeiert wird, nicht aber in
Friedrichs Familie. Dem christlichen \Cite{Lichtmeer der Liebe} stehen
\Cite{pestige Kellerhöhlen} entgegen, in denen sich Friedrich gleichnishaft
wähnt, da er seinem \Cite{großen Bruder}, dem heimatlosen Ahasver zu folgen
verurteilt sei. Der Verweis auf die Gestalt des zu ewiger Wanderschaft
verurteilten Juden ist zwar Indiz dafür, dass die \Cite{Ufer}, zwischen denen
sich Friedrich hin und her \Cite{taumeln} fühlt, durch die Grenze der
Glaubensgemeinschaft getrennt werden. Doch dieser Gegensatz wird sogleich
erweitert durch weitere Befremdlichkeiten, die die \Cite{heimatlose} Lage
Friedrichs verallgemeinert darstellen: Den \Cite{wohlarrangierten
  Familienbildern aus gesitteten Häusern} \SourceRef{II}{18} gilt seine
Abscheu ebenso wie der
ignoranten Bürgerlichkeit seiner Verwandten. Er will keinen \Cite{Brotberuf}
ergreifen, sondern bildender Künstler werden. Im verstorbenen Vater sieht er
einen verlogenen Patriarchen, der ihm seine \Cite{Jugend versperrt} habe. Der
Gottesdienst, zu dem die Mutter ihn drängt, ist in seinen Augen ein beengender
\Cite{Leutedienst} vor einem Gott, der zum \Cite{verknöcherten Richter}
verfälscht worden sei \SourceRef{II}{19}.

Die Adressatin all dieser Beschwerden über die Unerträglichkeit der
verschiedenen Kontexte der Herkunftswelt ist die Mutter. Sie habe seine Seele
wie ein \Cite{nacktes Kind} ausgesetzt und ihn zum Hass statt zur Liebe
erzogen.

Der Dialog, in dem die Mutter nur die Rolle einer Stichworte gebenden
Projektionsfläche für die \Cite{fiebernden} Tiraden ihres Sohnes einnimmt,
zeigt die völlige Distanzierung des Protagonisten von den sozialen Bindungen
seiner Herkunft: Die angestammte Religionsgemeinschaft wird genauso radikal
als Identifikationspunkt abgelehnt wie die Familie und deren soziale und
ökonomische Schichtzugehörigkeit. Die sittlichen Normen des bürgerlichen
Umfelds erscheinen in Friedrichs Repliken als ebenso marode und sinnentleert
wie die religiösen Handlungen des Gottesdienstes. Die religiöse Gemeinschaft
ist zur säkularisierten Schaubühne formaler Pflichterfüllung geworden, in der
\Cite{wirtschaftliches Fortkommen} wichtiger ist als das Heil der \Cite{Seele}
\SourceRef{II}{19}.

Die Hauptfigur tritt hier als ein Subjekt auf, das sich von allen nur
herkunftsmäßig begründeten Bestimmungen seiner Existenz zu distanzieren sucht,
weil diese ihre normative und sinnspendende Kraft verloren haben.  In diesem
Entäußerungsfieber zieht es sich sozusagen auf die abstrakteste Form seiner
Identität, seine \Cite{Seele}, zurück. Die dabei erfolgende Auflösung
angestammter Identifikationsschemata lässt sich analytisch durchaus mit dem
von Silvio Vietta vorgeschlagenen Begriff der \Cite{Ichdissoziation}
beschreiben, mit dem die \Cite{im Expressionismus zur Darstellung kommende
  grundlegende Strukturkrise des modernen Subjekts} zusammengefasst wird. Mit
Vietta gehe ich davon aus, dass es diese Grundlagenkrise ist, die zu den
\Cite{typischen expressionistischen Aufbruchs- und Erneuerungsversuchen}
geführt hat.\Footnote{\BibRef{vietta.94}{186}.}

Auch in der \Title{Wandlung} wird aus der Krise der handlungstragenden
Figur eine derartige Aufbruchsstimmung entwickelt: Um den Fokus der
Selbstrettung seiner Seele gruppieren sich bei Friedrich einige wenige positiv
besetzte Begleitmomente, die allesamt auf eine unbekannte, aber verheißungsvolle
neue Bezugswelt verweisen: Da sind die \Cite{Lichter}, die bei den
\Cite{Fremden} strahlen, und \Cite{Milde und Güte und Liebe}, die bei ihnen
\Cite{wächst} \SourceRef{II}{19}. Und schon bevor er seine Mutter mit
Vorwürfen überzog, hatte Friedrich zu sich gesagt:

\begin{BlockQuote}
  Zu denen drüben gehöre ich. Einfacher Mensch, bereit zu beweisen. Fort mit
  aller Zersplitterung. Nicht mehr länger stolz schützen, die ich
  verachte.
  \SourceRef{II}{17}
\end{BlockQuote}
Die Einordnung in eine neue, selbst gewählte Gemeinschaft soll also das
Heilmittel gegen die innere \Cite{Zersplitterung} und \Cite{Zerrissenheit}
darstellen. Damit wird die Identitätskrise, die sich in Form von Abscheu gegen
bestehende soziale Eingliederungen artikuliert hat, in ein Bedürfnis nach
neuer, höherwertiger Gruppenzugehörigkeit transformiert.\Footnote{Birgit
  Schreiber beschreibt dies auch als eine \Cite{Umdeutung} des Begriffs der
  Familie zur \Cite{Bruder- und Schwesterfamilie}, die auf die
  \Cite{antizipierte Menschengemeinschaft der Zukunft} verweist. Siehe
  \BibRef{schrei.97}{74}.}

\HeadingTwo{Das unbefriedigte Gemeinschaftsideal}

Friedrichs Bedürfnis nach Gemeinschaft darf nicht mit dem Wunsch nach
Geselligkeit verwechselt werden. Nachdem er sich von seiner Mutter getrennt
hat und dabei etwas \Cite{entzwei} \SourceRef{II}{19} gegangen ist, trifft er
auf seinen
\Cite{Freund}, der zugleich der Bruder seiner Geliebten ist. Friedrich verhält
sich abweisend und zynisch -- er will kein \Cite{Mitleid}, er \Cite{brauche
  niemand} -- und setzt so seine Abgrenzungshaltung auch auf dieser Ebene
fort. Rein verbal gibt er seiner Identitätskrise einen souveränen Anstrich:
\Cite{Ich bin allein stark genug, ganz allein.} \SourceRef{II}{20}.

Doch als der Freund erwähnt, dass man Freiwillige für \Cite{den Kampf gegen
  die Wilden} suche, wird die Ausrichtung von Friedrichs Zugehörigkeitsidealen
offenbar. Begeistert stürzt er sich auf die Gelegenheit, die er als ein
Weihnachtsgeschenk des \Cite{Vaterlandes} bejubelt:

\begin{BlockQuote}
  Nun kommt Befreiung aus dumpfer quälender Enge. Oh, der Kampf wird uns alle
  einen \ldots Die große Zeit wird uns alle zu Großen gebären \ldots Auferstehen
  wird der Geist [..] Nun kann ich meine Pflicht tun. Nun kann ich beweisen, daß
  ich zu ihnen gehöre.
  \SourceRef{II}{20-21}
\end{BlockQuote}
Der nur vorübergehend auf sich selbst zurückgezogene Kritiker beengender
Zugehörigkeiten agiert hier in bezeichnender Weise ein Gemeinschaftsideal aus,
das den Wunsch nach \Cite{Befreiung} ausgerechnet durch dienstbare
Eingliederung in einen Kriegsapparat und tätige Erfüllung einer (selbst
auserkorenen) \Cite{Pflicht} umsetzbar erscheinen lässt. Dass er durch diese
Unterordnung Anteil an \Cite{Geist}, \Cite{Größe} und \Cite{Schönheit}
\SourceRef{II}{21}
erwerben zu können glaubt, zeigt, wie sehr ihm die Welt des Vaterlandes zur
Projektionsfläche von Erlösungswünschen geworden ist.

Der \Cite{Suchende} befindet sich hier im Stadium der Ablösung und des Aufbruchs
vom Hergebrachten, ohne bereits eine tragfähige Selbstkonzeption entwickelt zu
haben. 
Das Identitätsdefizit soll aufgefüllt werden durch Unterordnung unter die
Sinngebung einer Gemeinschaft, die \Cite{Einheit} und Seelenreinigung
herzustellen verspricht. Die Eigenleistung des Subjekts beschränkt sich
darauf, seinen Subjektstatus an diese übergeordnete Instanz abzutreten und
sich in eben dieser funktionalen Subsumtion \Cite{beweisen} \SourceRef{II}{21}
zu wollen.

Es zeigt sich hierin das Leitmotiv der \emph{Einheit und Ganzheit in
  sinngebender Gemeinschaft}, das im Verlauf des Dramas in der (fehlgeleiteten)
Variante des kriegerischen Nationalismus vorgeführt und später zum Ideal der
\emph{Menschheitsgemeinschaft} generalisiert wird. Die visionsartigen
Darstellungen dieses Motivs sind in der \Title{Wandlung} durchgängig vom
Aspekt der allumfassenden \Cite{Liebe} begleitet.

Ernst Toller hat in Friedrichs Kriegsbegeisterung und Einheitsträumen
zweifellos ein gutes Stück jener Verbrüderungshoffnungen gestaltet, die er
selbst bei Beginn des ersten Weltkriegs verspürte:

\begin{BlockQuote}
  Der Kaiser kennt keine Parteien mehr [..] das Land keine Rassen mehr, alle
  sprechen eine Sprache, alle verteidigen eine Mutter,
  Deutschland.
  \SourceRef{IV}{50}\Footnote{Laut Walter Sokel bedeutete für Ernst Toller
    \Cite{der Krieg doch die heißersehnte Erlösung aus seiner Isolation als Jude,
      Erlösung vor allem aus seiner Isolation als Mensch, der unverstanden und
      vereinzelt als funktionsloses Atom im modernen Gesellschaftsmechanismus
      suchend irrte.}. Siehe \BibRef{sokel.81}{25}.}
\end{BlockQuote}
In der \Title{Wandlung} ist das Gemeinschaftsideal als Verbindung von
Zugehörigkeitswünschen und der Sehnsucht nach Seelenheil und umfassender Liebe
gekennzeichnet.\Footnote{Zur Verknüpfung von Nation und Religion
  vgl. \BibRef{schrei.97}{85}.}  Mit dieser Wunschvorstellung kontrastieren
Friedrichs aggressives Verhalten, die rigorose Kritik an der schockierten
Mutter, seine verletzenden Vorwürfen und zynischen Provokationen. In der
emphatischen Verknüpfung von Kriegsbeginn und \Cite{Abend der Liebe}
\SourceRef{II}{21}
konzentriert sich die verklärte Dissonanz der initialen
Aufbruchsszene.\Footnote{\Abr{Vgl} \BibRef{benson.87}{38}: \Cite{Toller hat
    für diese Szene bewusst die Zeit der Weihnacht, Symbol für Friede und
    Versöhnung, gewählt, um Friedrichs Fehlentscheidung zu betonen.}}

\HeadingTwo{Die kriegführende Nation als inhumane Gemeinschaftsfiktion}

Im zweiten bis sechsten Bild des Dramas wird Friedrichs Teilnahme am Krieg und
dessen Leidhaftigkeit, Einförmigkeit und unerbittliche Systematik
gezeigt. Friedrichs Integrationswille wird zunehmend als Irrweg vorgeführt,
indem die Defizite seines Vaterlands-Ideals aufgedeckt werden.

So ist schon das zweite Bild eine \Cite{traumferne} Momentaufnahme der
qualvollen, wie endlos empfundenen Kreisläufe des Soldatenlebens und
-sterbens. Die namenlosen Insassen eines militärischen Transportzuges, einer
davon totenköpfig, ein anderer mit dem \Cite{Antlitz Friedrichs}, verwünschen,
dass sie überhaupt geboren wurden und nun als \Cite{ewig geängstigte Kinder}
\Cite{ewig fahren} und \Cite{ewig verwesen} müssen \SourceRef{II}{21-22}. 
Unmittelbar nach dem euphorischen Aufbruch Friedrichs wird so ein düsterer
Kontrapunkt zu seiner Kriegsbegeisterung gesetzt.

Im dritten Bild, das wieder Teil der eigentlichen Handlungssequenz ist, wird
Friedrichs Ideal von der \Cite{einenden} Kraft des Kampfes explizit
angegriffen: Anderen Soldaten ist klar, dass die Gemeinschaft namens
\Cite{Vaterland} durch Ausbeutung und Machtunterschiede gekennzeichnet
ist. Sie verhöhnen Friedrichs Integrationshoffnungen,\Footnote{\LongSourceRef{II}{24}:
  \Cite{Und wenn du tausendmal in unseren Reihen kämpfst, darum bleibst du doch
    der Fremde.}}  worauf er nur mit trotzigen Bekenntnissen seines Heldenmuts
reagieren kann.\Footnote{Birgit Schreiber sieht darin, \Cite{den paradoxen
    Versuch, durch Hervorhebung und Auszeichnung [..] \Quote{Gleichheit} unter Beweis zu
    stellen}. Siehe \BibRef{schrei.97}{26}.}

In der Vaterlandskritik, die in dieser Szene laut wird, werden \emph{ex
  negativo} konstitutive Merkmale des Gemeinschaftsideals der Autornorm
thematisiert: Sie soll keine inneren Gegensätze aufweisen, darf nicht durch
Ungleichheit oder gar Ausnutzungsverhältnisse verfälscht sein und muss den
gutwilligen, tätigen Mitstreiter ohne Ansehen der Herkunft in sich
aufnehmen.  Es zeichnet sich ab, dass die nationale Gemeinschaft (auch als
Kampfgemeinschaft) diese Eigenschaften nicht aufweist und \Cite{der Krieg
  nicht die Überwindung von Klassen- und Gruppengegensätzen, sondern ihren
  Fortbestand bedeutet}\Footnote{\BibRef{rothst.87}{51}.}.

Das fünfte Bild zeigt Friedrich als Kranken im Lazarett. In Fieberträumen
phantasiert er die Fortsetzung seiner wandernden Suche nach Identität, nach
\Cite{steinigen Gipfeln}, die er aber vor lauter \Cite{Wüste} nicht erreichen
könne \SourceRef{II}{28}. Im Traum erscheint ihm auch wieder die
Ahasver-Gestalt, doch mit ihm
will Friedrich nicht mehr wandern. Die Szene ist durchdrungen von einer
Kreuz-Symbolik, die die Verkehrung christlicher Ethik im Kriegskontext zur
Geltung bringen soll: Die Lazarettschwester, die das rote Kreuz trägt, wird
von Friedrich mit der \Cite{Mutter Gottes} in Verbindung gebracht. Er glaubt,
dass sie als \Cite{Kreuzträgerin, Verkünderin der Liebe} alle Kranken und auch
die Feinde \Cite{draußen} heilen müsste, und ist enttäuscht, dass sie sich
stattdessen über eine Siegesmeldung freut, derzufolge der Feind
\Cite{zehntausend Tote} zu beklagen habe \SourceRef{II}{29}. Friedrich selbst
erhält als
\Cite{Held} von einem Offizier für seinen Einsatz \Cite{das Kreuz} als
Auszeichnung und zugleich \Cite{Bürgerrechte}\Footnote{\Abr{Vgl} \BibRef{schrei.97}{88}:
  \Cite{Der Erwerb der Bürgerrechte gestaltet sich formal identisch mit dem
    Eintritt in eine Religionsgemeinschaft.} Dies ergibt sich für Schreiber aus
  der religiösen Konnotation des überreichten Kreuzes.}: \Cite{Sieg stürmt ins
  Land, Sie gehören zu den Siegern.}  Doch Friedrich ist verunsichert:

\begin{BlockQuote}
  Durch zehntausend Tote gehöre ich zu ihnen. [..] Ist das Befreiung? Ist das
  die große Zeit? Sind das die großen Menschen?
  \SourceRef{II}{29}
\end{BlockQuote}
Das Ziel, in die Gemeinschaft des Vaterlandes aufgenommen zu werden, ist
erreicht. Doch die Sinnhaftigkeit und \Quote{Größe} der todbringenden
Veranstaltung steht in Frage. Die von Friedrich hergestellte Überblendung von
Krieg und christlicher Liebe, die in der Aufbruchsszene am Dramenbeginn als
positive Perspektivierung seiner Eingliederungsbestrebungen ihren Anfang nahm,
ist angesichts von Tod und Zerstörung nicht mehr haltbar. Dem Wunschbild
von der Neugeburt als \Cite{großer Mensch}\Footnote{\Abr{Vgl} \SourceRef{II}{21}:
  \Cite{Die große Zeit wird uns alle zu Großen gebären.}}, dessen innere
\Cite{Zerrissenheit} durch Beteiligung am \Cite{einenden} Kampf aufgehoben
werden sollte, steht die Blutigkeit seiner Zugehörigkeitsbeweise entgegen, die
jedes heilspendende Kreuz gleichsam rot färben.

\HeadingTwo{Der Tod als Feind des Geistes}\label{TOD}

Es wurde bereits erwähnt, dass in den Personalangaben des Dramas der \Cite{Tod
  als Feind des Geistes} aufgeführt wird, der in Gestalt verschiedener
totenköpfiger Figuren auftritt. Mit der Analyse dieses allegorischen
Prinzips soll im Folgenden ein zentraler konzeptioneller Antagonismus des
Dramas herausgearbeitet werden, der die bereits dargestellte Kritik der
vaterländischen Kampfgemeinschaft in einen systematischen Zusammenhang
stellt.

Im allegorischen Vorspiel \Cite{Die Totenkaserne} tritt der Tod in einer
Doppelrolle auf: Der \Cite{Kriegstod} führt dem Kollegen \Cite{Friedenstod}
sein strammes Regiment über die nach militärischem Rang geordneten Kriegstoten
vor und lässt Skelette von einfachen Soldaten -- die \Cite{schlichten Nummern} --
unter Aufsicht von toten Offizieren exerzieren \SourceRef{II}{13}.  Den
Friedenstod kann dieses
\Cite{ordnende Prinzip} \SourceRef{II}{14} nur vorübergehend beeindrucken. Er
erkennt, dass die
Ordnung bloß das übernommene System des Krieges ist, und verhöhnt die
Unterwürfigkeit des Kollegen:

\begin{BlockQuote}
  Vor mir ist jeder gleich. Doch Ihr Prinzip ist nicht von unsrer Welt. Der
  Krieg hat Sie geschlagen.
  \SourceRef{II}{15}
\end{BlockQuote}
Der Tod wird damit qualitativ charakterisiert: In seiner eigentlichen Form,
dem Friedenstod, gilt er als nivellierender, geradezu gerechter Weg ins
Jenseits, der sich nicht durch die \Cite{Vorurteile} und unmenschlichen
\Cite{Ordnungen} des Diesseits manipulieren lässt: Vor ihm ist jeder
gleich. Der Kriegstod wirkt dagegen als lächerlicher \Cite{Tod von heute}, der
seine Souveränität an das \Cite{Kriegssystem} abgetreten hat
\SourceRef{II}{15-16}.

Der Topos vom Tod als Eintritt in ein nivellierendes Jenseits, in dem
diesseitige Unterschiede und Gegensätze aufgehoben sind, wird in einem der
\Cite{traumfernen} Bilder, die Friedrichs Fronterlebnisse umranken, praktisch
illustriert: In einem tanzenden Reigen finden sich Skelette von Kriegstoten
zusammen, die \Cite{nicht mehr Freund und Feind} sondern nun \Cite{alle
  gleich} sind \SourceRef{II}{26}. In ihre Mitte nehmen sie ein zu Tode
geschändetes Mädchen auf,
denn \Cite{aus ist's mit der Scham} \SourceRef{II}{27} in dieser tanzenden
Gemeinschaft von Geschändeten. Die hier einträchtig tanzenden Toten sind dem
Kriegssystem
entronnen. Sie feiern eine Gleichheit, die in der Logik des Dramas als Idylle
gelten könnte, wenn sie nicht dem Tod zu verdanken wäre.

Dass der Tod aber nur die schlechte Erlösungsoption eines inhumanen Diesseits
ist, wird in einer weiteren \Cite{traumartigen} Szene unterstrichen. Sie zeigt
\Cite{die Krüppel} in einer militärischen Bettenstation. Ein totenköpfiger
Medizin-Professor lässt seine \Cite{Musterexemplare} im Stechschritt
aufmarschieren \SourceRef{II}{30}: Mit mechanischen Gliedmaßen ausgestattete
menschliche
\Cite{Fleischrümpfe}, die durch die Medizintechnik wieder Befehlen gehorchen
und sogar Nachwuchs zeugen können, werden von ihm stolz als wissenschaftliche
Errungenschaften vorgeführt:

\begin{BlockQuote}
  Wir Vertreter der Synthese, / Die Rüstungsindustrie geht analytisch vor -- /
  Die Herren Chemiker und Ingenieure / Sie mögen ruhig Waffen schmieden / Und
  unerhörte Gase fabrizieren, / Wir halten mit.
  \SourceRef{II}{30}
\end{BlockQuote}
Die Demonstration wird unterbrochen, als ein \Cite{Hörer} mit dem
\Cite{Antlitz Friedrichs} in Ohnmacht fällt \SourceRef{II}{31}. Die technische
Vorführung des
Menschenmaterials wird dann abgelöst durch einen Reigen der Qual, einen
\Cite{Mischchor} \SourceRef{II}{32} der bettlägerigen Schwerstverletzten, die
sich als Gefangene
ihrer zerstörten Körper artikulieren. Ein Pfarrer, der ebenfalls das
\Cite{Antlitz Friedrichs} \SourceRef{II}{33} trägt, verliert vor diesem Elend
seinen Glauben und
sinkt in sich zusammen. Genauso ergeht es einer Gruppe Schwestern, deren
heilungswillige \Cite{Nächstenliebe} von den nach Sterbehilfe verlangenden
Krüppeln als \Cite{Flickwerk} \SourceRef{II}{34} abgelehnt
wird.\Footnote{\BibRef{bebend.90}{42}:
  \Cite{So werden Glaube und Karitas als wertlose Prinzipien verabschiedet.}}

Der Tod ist hier der ersehnte Ausweg, neben dem das Angebot religiös
motivierter Heilung zur Kompensationslüge verkommt und -- mit dem
\Cite{Antlitz Friedrichs} -- ohnmächtig zusammenbricht. Die Gestalt des
totenköpfigen Professors verkörpert den \Cite{Tod als Feind des Geistes}, indem
er für ein mit wissenschaftlicher Präzision arbeitendes System kriegsmäßiger
Funktionalisierung steht, das seinem Menschenmaterial den Subjektstatus nicht
nur ideell, sondern ganz konkret physisch entzieht und damit Todeswünsche als
Erlösungshoffnungen produziert. Es wird damit gezeigt, dass die kriegführende
Nationalgemeinschaft den von Friedrich eigentlich angestrebten Idealen von
Gleichheit, Liebe, innerer Erlösung und Seelenheil nur eine armselige
Perspektive bieten kann -- nämlich den Tod.

In der Logik der Autornorm ist der verbrüdernde, revolutionäre \Cite{Geist}
als die wahre Kraft der Erlösung vorgesehen. Der Tod ist \Cite{Feind des
  Geistes}, wenn er -- in Verdrehung der Heilsidee des Autors - zum Träger
dieser Erlösungsfunktion gemacht wird. Das Kriegssystems ist im Sinne
dieser Erlösungsalternativen das System des Todes, das den potentiellen
\Cite{Geist} des Protagonisten blockiert und damit die -- von Toller gemäß der
Revolutionsauffassung Gustav Landauers verstandene -- intersubjektive Kraft,
die wahre Gemeinschaft herzustellen vermag, an ihrer Entfaltung hindert.

Der als Akteur des Todessystems dargestellte Wissenschaftler repräsentiert
aus dieser Perspektive die pervertierte Anwendung geistigen Potentials in Form
einer naturwissenschaftlich-technischen Intelligenz, die sich der Sphäre der
staatlichen Macht unterordnet. Die kulturkritische Norm menschengemäßer
Anwendung des \Cite{Geistes} impliziert somit auch eine spezifische Frontstellung
gegenüber Wissenschaft und Technik.

Die Befreiung Friedrichs aus dem System des Todes und die Lösung seines
\Cite{Geistes} aus der Unterordnung unter die Macht ist die logische
Verlaufsform, in der dieser zentrale Antagonismus der
\Title{Wandlung} aufgelöst wird.

\HeadingTwo{Menschlichkeit als Offenbarung}

Während der erste Teil des Dramas die fortgesetzte Sinnkrise Friedrichs auf
seinem vaterländischen Irrweg bebildert, ist im Mittelteil die eigentliche
\Title{Wandlung} dargestellt. Der Protagonist -- nach dem Krieg als
schaffender Künstler tätig -- wird zum Messias im Sinne der
Verkündungsabsicht des Autors gemacht.  Die endgültige Abkehr vom
\Cite{Vaterland} wird im siebten Bild durch zwei letzte desillusionierende
Begegnungen besiegelt: Friedrichs Geliebte sagt sich von
ihm los, weil er aufgrund seiner Herkunft für ihren Vater inakzeptabel ist und
sie ihre zu erbende \Cite{Scholle} \SourceRef{II}{36} nicht aufgeben will.
Eine hausierende Kriegsinvalidin, die ihren verkrüppelten Mann wie halb geschlachtetes
\Cite{Vieh} mit sich führen muss, macht schließlich klar, dass das
\Cite{Vaterland} nur ein Deckmantel sei für die Kriegstreiberei von \Cite{Reichen,
  die prassen} \SourceRef{II}{38} und sich des Segens der \Cite{Kupplerin
  Kirche} \SourceRef{II}{39} bedienen. Die
Zerstörung des diskreditierten Ideals findet statt als konkretes Zertrümmern eines
\Cite{Symbol[s] [..] des siegreichen Vaterlandes} \SourceRef{II}{35}, das
Friedrich in seinem Bildhauer-Atelier bearbeitet hatte.

Ohne Vorgabe eines höheren Lebenssinns ist Friedrich wieder identitätslos und
kann die innere Zerrissenheit nicht bewältigen. Seinen Selbstmord kann
nur ein \emph{deus ex machina} verhindern. \Cite{Die Schwester} ist jene
Mittlerfigur, die Friedrich den rechten Weg zeigt. In konsequenter
Generalisierung des abstrakten Gemeinschaftsideals, das sich in Friedrichs
Suche schon andeutete, weist sie ihm einen Weg \Cite{hinauf}, der \Cite{auch
  zur Mutter} und \Cite{auch zu deinem Land} führt \SourceRef{II}{40} -- also
die Welt der
persönlichen Herkunft und des (neutralisierten) Vaterlandes mit einzuschließen
verspricht.  In einer Gleichsetzungskette wird die Richtung des Weges
bestimmt: Friedrich soll sich \Cite{zu Gott} bewegen, der \Cite{Geist und
  Liebe und Kraft ist} und \Cite{in der Menschheit lebt}. Deswegen soll er
\Cite{zu den Menschen} gehen. Dem innerlich verzweifelten, \Cite{heimatlosen}
Subjekt wird die Hinwendung zu den Mitmenschen als universeller Weg empfohlen,
der \Cite{isolierte Held kann nur durch die Erlösung der ganzen Welt seine
  eigene erwirken}.\Footnote{\BibRef{sokel.81}{26}.}

Die Bestimmung des rechten Weges nimmt so die reine \emph{humanitas} zur
Richtschnur und basiert auf dem Glauben an eine urtümliche, göttliche Kraft,
die jedem Menschen innewohnt und nur geweckt werden muss. Dies ist das Credo
des expressionistischen \Quote{O Mensch}--Kultes und artverwandter Theorien wie
dem bereits erwähnten Sozialismusverständnis Gustav Landauers. Es ist dieser
Glaube an ein verschüttetes, eigentliches \Quote{Wesen} des Menschen, das die
expressionistischen Visionen vom \Quote{Neuen Menschen} nährte, der durch
\Cite{Wiedergeburt} und \Cite{Erweckung} zu schaffen sei.  Die
spezifisch messianische Form dieser Erweckungsideologie, die in der
\Title{Wandlung} entwickelt wird, unterscheidet zwischen den wenigen
\Quote{Geistigen}, die das Potential zum Erwecker in sich tragen, und dem
großen Rest der Menschheit, der innerlich unbewusst auf die Erlösung wartet.
Der Protagonist der Wandlung ist insofern ein besonderer Fall, als er
sozusagen über Umwege zur messianischen Gestalt wird und selbst erst einen
aufwendigen Weg der Läuterung durchlaufen muss, der ihn von den Fehlern seines
Irrweges reinigt und seine Wiedergeburt als wahrer Mensch sicher stellt.  So
lautet der Hinweis der Schwester: \Cite{Wer zu den Menschen gehen will, muss
  erst in sich den Menschen finden.} \SourceRef{II}{40}.

Die Abstraktion der \emph{reinen Menschlichkeit} ist eine Formel, die die
individuell zu leistende Aufgabe der Identitätsbildung moderner Subjekte
verwirft, indem die gesellschaftliche Ausdifferenzierung inklusive
aller Formen ökonomischer, politischer und funktionaler Relationierung der
sozialen Teilsysteme ideell zurückgenommen wird.  Das Problem sozialer
Isolation hebt sich in dieser Abstraktion auf, man ist \Cite{allein, und doch
  mit allen} \SourceRef{II}{40}.  Scheinbare normative Autonomie erhält das
Subjekt, das \Cite{selber Angeklagter, selber Richter} sein
soll, durch Zusprechung einer inneren Kraft, die nicht begründet, sondern
(primär den Künstlern) als mystische \Quote{Gabe} zugewiesen wird.

Für die qualitative Füllung dieser begrifflichen und ideologischen Hülse
ergeben sich in der zweiten Hälfte des Dramas noch gewisse Hinweise und
Ausdeutungen. Im Kern bleibt es aber bei der Fokussierung auf das vage
Konzept der reinen Menschlichkeit. Der Reinigung des fehlerhaften
Führer-Subjekts folgt der Vollzug der messianischen Verkündungsrevolution.

\HeadingTwo{Religiöse Anleihen des Menschheitsmythos}

Die Transformation Friedrichs zur Messiasfigur wird in Traumszenen dargestellt, die
den Menschheitsmythos mit christlichen Motiven anreichern. Seine
\Cite{Wiedergeburt} ist eine Kombination aus \Quote{Selbstkreuzigung},
lichtumstrahlter Kindsgeburt und Auferstehung \SourceRef{II}{43-46}.
Die christliche Passionsgeschichte wird dabei zum vorbildlichen Rezept
aktiver Selbstreinigung modifiziert: \Cite{Nicht Römer schlugen ihn ans Kreuz
  / Er kreuzigte sich selbst.}
\SourceRef{II}{44}\Footnote{\BibRef{reimer.00}{50}: \Cite{Christus
    wird nicht verstanden als der Sohn Gottes, sondern als Mensch, der durch
    seinen Opfertod Verantwortung übernimmt.}}.  So wird Friedrichs Auslöschung
seines alten, verfälschten Ichs motiviert, dessen \Cite{Schmach} er nach
seiner Selbstkreuzigung und Auferstehung \Cite{wie Dornenkronen}
ablegen kann \SourceRef{II}{46}.  Sein Verhalten stilisiert ihn fortan zum
weltlichen Messias: Mit gütiger, selbstloser Liebe behandelt er alle, die ihm
begegnen. Sein
\Cite{Mitleid} lässt \Cite{Kranke} aus Wahnphantasien \Cite{erwachen}
\SourceRef{II}{56}. Seine friedliche Gleichbehandlung gilt denen, die ihn zu
\Cite{hassen} \SourceRef{II}{53} glauben, ebenso wie verschiedenen
Frauengestalten, die seinen Körper statt seiner
\Cite{Güte} \SourceRef{II}{57} begehren.\Footnote{\Abr{Vgl} auch
  \SourceRef{II}{52}: \Cite{\DramName{Friedrich}: Ich will nicht deine
    Umarmung. Gab ich jedem Recht auf meinen Körper? [..] Armes Weib!
    Ungelöste.}
  \BibRef{kim.98}{81}, kommentiert: \Cite{Die an ein
    Individuum gerichtete erotische Liebe ist ein Hindernis dafür, die Freiheit,
    \AbrPair{d}{h} den Geist, zu erlangen.}}

Die inspirierte Stellung der Messiasgestalt zeigt sich in seinem Wissen um die
Zukunft, der magisch mitreißenden Kraft seiner Rede und der Fähigkeit, die
Lage der Menschheit überindividuell zu erfassen: \Cite{Ihr Brüder und
  Schwestern: Keinen von euch kenne ich und doch weiß ich um euch alle.}
\SourceRef{II}{58}. Die
visionäre Predigt des \Cite{Führer[s]} eröffnet seinen
Zuhörern im kultisch totalisierten Gemeinschaftsideal das Tor zur einenden
\Cite{Menschheitskathedrale} \SourceRef{II}{51}.\label{mkath}

Der Grund für diese pseudoreligiöse Überhöhung ergibt sich aus dem Grundthema
des Dramas. Was in der \Title{Wandlung} als Problemkomplex der
\Cite{Zerrissenheit}, \Cite{Heimatlosigkeit} und Sinnsuche gestaltet wird und
im universellen Menschheitsideal seine Heilslehre und Erlösungsperspektive erhält,   
ist implizit ein Bewältigungsversuch von säkularen \Quote{Auflösungsprozessen}.  
Die Ungewissheit, wie \Cite{der Mensch} trotz der empfundenen Säkularisierung
und Funktionalisierung der angestammten Religionsysteme sein \emph{Seelenheil}
im Diesseits erlangen kann, evoziert ein Bedürfnis nach (neu zu schaffender)
Verbindlichkeit und Totalität. 
\Cite{Der Mensch}, also eigentlich das all seinen individuellen
Bestimmungen entledigte moderne Subjekt, wird aus diesem Bedürfnis heraus zum
neuen Heiligtum erhoben. Weil damit die Wiedererlangung einer ganzheitlichen
Weltsicht angestrebt ist und am Prinzip der Totalität festgehalten wird, kann
der Glaube an die dem Menschen innewohnenden Selbstheilungskräfte gar nicht zur
Selbstreflexion und subjektautonomen Identitätsbildung unter Anerkennung der
Pluralität der modernen Welt führen. 
Die postulierte Gleichheit aller Menschen unter gleichzeitiger Hervorhebung
gewissser begabter Priestergestalten wird vage aus inneren Urzuständen
hergeleitet. Das Dilemma der menschelnden Heilslehre liegt somit darin, dass
sie ihren universellen Verbindlichkeitanspruch nicht aus explizierender
Begründung beziehen kann und deswegen dem Bestand der religiösen Tradition
doch gewisse mystische Extrakte und Analogien \Quote{abzapfen} muss, um nicht völlig
unvermittelt daher zu kommen.

\HeadingTwo{Menschlichkeitskult und Revolution}

Friedrichs Heilsweg, den die zweite Dramenhälfte zeigt, vollzieht sich als eine
Annäherung an die Welt des (proletarischen) \Cite{Volkes} \SourceRef{II}{58}.  
Als \Cite{Schlafbursche} \SourceRef{II}{41}
bei einer Arbeiterfamilie zeigt er letzte Zeichen von Egoismus, wird
dann zur selbstreinigenden \Cite{Arbeit} in die \Cite{große Fabrik} geführt 
\SourceRef{II}{42,43}
und kennt als auferstandener \Cite{Wanderer} schließlich den \Cite{Weg zur
  Arbeitsstätte} \SourceRef{II}{46}. 
Die anvisierte Sphäre des Dialogs mit den Massen wird in
Versammlungsszenen und tableauartigen Begegnungssequenzen dargestellt, in
denen der Wissende zu den Menschen spricht. Das grundlegende Thema ist das
Leiden des Volkes, sein Hunger und seine Irreführung. Die herrschenden Kräfte
der Gesellschaft -- Kriegsveteranen, Wissenschaft und Kirche -- treten in Gestalt
hohler Phrasendrescher auf, die die Bedürfnisse des Volkes
unterdrücken. Friedrich ist dagegen der bemühte, mitfühlende Vertreter der
wahren Volksfürsorge, sein revolutionäres Prinzip ist die verbrüdernde
Einfühlung.

Die religiösen Konnotationen des Menschlichkeitskultes, die im Seelenheil
ihren Fokus haben, überlagern dabei politische und soziale Stellungnahmen der
Führergestalt. In den Szenarien des Dialogs zwischen Volk und Messias gibt es
nur Leidende oder (falsche) Verführer, die von der visionären Kraft Friedrichs
erleuchtet bzw. in ihre Schranken verwiesen werden.

Die Mechanik der revolutionären Entwicklung bricht sich rhetorisch Bahn. Die
Versprühung von \Cite{Geist} verläuft als verbalisierte Anteilnahme des
Führers am lauschenden Publikum. Das zugrundeliegende Schema ist Gustav
Landauers Erweckungssozialismus:\label{LANDAUER}

\begin{BlockQuote}
  Der Einzelne, über den es [das Ideal] wie eine Erleuchtung kam, sucht sich
  Gefährten [..] Geist ist Gemeingeist. [..] ist Menschenbund. [..] Aus den
  Herzen der Einzelnen bricht dieses unbändige Verlangen in gleicher, in
  geeinter Weise heraus; und so wird die Wirklichkeit des Neuen geschaffen
  [..]\Footnote{Gustav Landauer, \Cite{Aufruf zum Sozialismus} (1911), zitiert
    nach \BibRef{ruehle.73}{904-905}. Fritton merkt an, das
    \Cite{Wechselverhältnis zwischen Geist und Gesellschaft [bleibe] bei Toller
      wie bei Landauer im mystischen Dunkel}. Allerdings habe Toller konkretere
    politische Aussagen Landauers wohl nicht zur Kenntnis genommen, so \AbrPair{z}{B}
    dessen Konzept revolutionärer Landeroberung. Siehe \BibRef{fritto.86}{137} und
    141.}
\end{BlockQuote}
In dieser Weise verläuft die finale Szene kollektiven revolutionären
Aufbruchs. Friedrichs Rede bringt dem Volk die Erkenntnis: \Cite{Daß wir es
  vergaßen! Wir sind doch Menschen!} \SourceRef{II}{60}.
Die soziale Revolution, die Ausweg aus
einer Welt des Elends sein will, dessen strukturelle Merkmale nur
gleichnishaft und wie zur Illustration angesprochen werden, ist alles andere
als Klassenkampf. Ein Volksredner, der das Volk zum gewalttätigen Aufstand
anstacheln wollte, wird von Friedrich als verlogener Aufhetzer enttarnt, der
im Volk die \Cite{Masse}, aber nicht den \Cite{Menschen} sehe. Die Revolution
funktioniert in der \Title{Wandlung} als Verbrüderungskult, der keine
politischen Gegner kennt:

\begin{BlockQuote}
  Geht hin zu den Reichen und zeigt ihnen ihr Herz, das ein Schutthaufen
  ward. Doch seid gütig zu ihnen, denn auch sie sind Arme, Verirrte. 
  \SourceRef{II}{61}
\end{BlockQuote}
Materielle Mängel und Schrecken der zu ändernden Welt werden von Friedrich nur
mit wenigen, verstreuten Formulierungen benannt: Neben \Cite{Armut und Elend}
\SourceRef{II}{50}
in den \Cite{Abfallgruben verpesteter Städte} \SourceRef{II}{55} ist die Rede von
\Cite{gewaltigen Maschinen} und \Cite{Eisenhäusern, von Rost zerfressen}
\SourceRef{II}{59}, die
die Menschen einengen. Doch eine differenzierte Kritik gesellschaftlicher
Verhältnisse ist damit nicht gemeint. Die Defizite der Welt kommen in der
Perspektive der \Title{Wandlung} stets nur als vergessene, verlorene
Menschlichkeit in den Blick.\Footnote{Tatsächlich hat das \Quote{richtige}
  Bewusstsein für Friedrich sogar Vorrang vor der leiblichen
  Zufriedenstellung. Siehe \SourceRef{II}{50}: \Cite{Ich aber will, daß ihr
    den Glauben an den Menschen habt, ehe ihr marschiert. Ich aber will, daß
    ihr Not leidet, so ihr ihn nicht besitzt.} \Abr{Vgl}
    \BibRef{reimer.00}{49}: \Cite{Im Vordergrund steht nicht die Beseitigung
      materiellen Elends, sondern die Erweckung von Menschlichkeit.}}

\HeadingTwo{Zusammenfassung und Würdigung}

Wie gezeigt werden konnte, ist Tollers \Title{Wandlung} als propagandistische
Antwort auf individuelle Identitätskrisen und Sinnverluste konzipiert, die die
Isolation des Einzelnen zusammen mit dem Elend der Welt aufzulösen verspricht.
Die krisenhaften Erfahrungen des exemplarischen Subjekts sind als die typisch
modernen Auflösungserscheinungen der normativen Verbindlichkeit von sozialen
Herkunftskontexten dargestellt. Der aus der Säkularisierungstendenz
entstehende Wertverlust der religiösen Sphäre wird als Ursache für ein
unbefriedigtes Bedürfnis nach Seelenheil ausgestaltet. Die
Orientierungsdefizite des auf sich selbst verwiesenen Subjekts sind im
strukturgebenden Bild der \emph{Sinnsuche} das wesentliche Movens der
Dramenhandlung.

In der Ausgangsszene des Dramas wird dieser Problemkomplex bereits vollständig
exponiert. Die Lösungskonzeption des Stückes, die explizit als Autornorm
gekennzeichnet ist, wird über die Demonstration einer falschen Alternative
vorbereitet: In der kriegführenden Nationalgemeinschaft ist Erlösung und
Seelenfrieden nur im Tod vorgesehen. Der Zweischritt des Dramas lässt der
Entlarvung des Falschen die Verkündung des Wahren folgen und entwickelt seine
Heilslehre als praktizierte Vision.

Die Auseinandersetzung mit moderner Welt ist in der \Title{Wandlung} eine
Leidensbewältigung doppelter Art: Den inneren Kern bilden Seelenprobleme von
Mensch und Mitmenschen, das äußere Szenario ist Kriegshorror und
Volkshunger.\Footnote{\Abr{Vgl} \BibRef{buetow.75}{66}: \Cite{Nach ihrer
    dramaturgischen Funktion beschränken sich der Krieg und das soziale Elend
    darauf, Quellen des Leids zu sein, die ihrerseits nach irgendwelchen Ursachen
    nicht befragt werden.}}  Die Kritik dieser nationalistischen Inhumanitäten
bereitet das Feld für die Offenbarungen des Menschheitskults.

\HeadingThree{Kritik nationalistischer Inhumanität}

Die Welt, die Toller in seiner \Title{Wandlung} verarbeitete, war nur
bedingt \Cite{modern} geprägt. Zwar sind die sozialgeschichtlichen Prozesse
der Modernisierung zu Beginn des \Nth{19} Jahrhunderts in vollem Gange, doch das
politische System und die vorherrschenden mentalen und normativen Strukturen
müssen wohl als konservativ und autoritär bezeichnet werden.  Die Absolutheit
nationalistischer Kriegslogik reduzierte darüber hinaus die \Cite{Autonomie}
gesellschaftlicher Teilsysteme auf ein funktionales Minimum. Im Kriegszustand
muss insgesamt von einer stärkeren normativen Verbindlichkeit der nationalen
Ideologie augegangen werden als in den \Quote{pluralistischeren} 
Friedenszeiten. 

Tollers in der \Title{Wandlung} ausgestaltete Kriegskritik setzt zwar gerade
nicht an dieser \Cite{Antimodernität} des Vaterlandsgedanken an, doch es muss
gewürdigt werden, dass seine Verbrüderungsvisionen den zeitgenössischen
Kampfesideologien ihr Humanitätsdefizit energisch vorhielt:

\begin{BlockQuote}
  Man muss sich das Zeitkolorit vergegenwärtigen, muss sich das Vorherrschen der
  chauvinistisch-rassistischen Denkformen nicht nur in Deutschland, sondern auch
  in den Nachbarlänern vor Augen halten, um das objektiv Revolutionäre einer
  Denkform würdigen zu können, die auf die Gemeinsamkeit der Menschen und ihre
  friedliche Koexistenz pochte.\Footnote{\BibRef{vietta.94}{201}. Nach Ansicht
    Silvio Viettas ist die ästhetische Moderne insgesamt durch einen
    \Cite{weltbürgerlichen} Zug gekennzeichnet, der einer \Cite{zunehmenden
      Nationalisierung und Totalisierung der Macht} entgegen wirke. Siehe
    \BibRef{vietta.98}{546}.}
\end{BlockQuote}
Auch die Funktionalisierung von Religion und Wissenschaft unter dem Regiment
von Militär und Obrigkeit wird in der \Title{Wandlung} treffend, wenn auch
holzschnittartig, kritisiert.\Footnote{In der Darstellung der Frontereignisse
  wird \AbrPair{z}{B} die kompensatorisch-funktionale Rolle der Lazarettschwestern und
  Militärpfarrer heraus gestellt und im sechsten Bild die \Cite{komplementäre}
  Ergänzung von Medizintechnik und Rüstungsindustrie sarkastisch gezeichnet. }
Es ist allerdings stets die Perspektive der \emph{eigentlichen} Verpflichtung für das
(innere) Menschenheil, aus der heraus die Akteure des Christentums, der
Medizin und der akademischen Sphäre in ein schlechtes Licht geraten. Kernpunkt
der Diskreditierung des Kriegssystems ist seine falsche Erlösungsoption:
\Cite{Der Tod als Feind des Geistes} befördert leidende Menschen ins Jenseits
und steht als ein \Cite{das Individuum wie die Gesamtkultur zerstörender
  Prozeß}\Footnote{\BibRef{vietta.98}{544}.} der Erweckung und Verbrüderung im
Diesseits diametral entgegen. 

In den Versammlungsszenen sind die Figuren, die die Führungsschicht der Nation
repräsentieren, rücksichtslos und ignorant dargestellt. Ein Professor betont,
die Wissenschaft sei stolze \Cite{Dienerin unseres Staates}
\SourceRef{II}{47}, ein Pfarrer preist Gott als \Cite{Herr[n] der Heerscharen}
\SourceRef{II}{48} und beide empfehlen dem Volk,
seinen Hunger weniger wichtig zu nehmen. Das Volk ist dagegen allein durch das
Merkmal des Leidens charakterisiert. In Tollers negativem, kritisierten Schema
vom \Cite{Vaterland} füllen die Menschen nur die Rolle der leidenden Masse
aus, sie sind Material, sind \Cite{Vieh}. Die simple Gegenüberstellung
dokumentiert eine Führer-Volk-Vorstellung, die auch in den Szenen des
Erweckungsvorgangs erhalten bleibt und dort die besondere Rolle des wissenden
Intellektuellen sichert.

Tollers Kritik des vaterländischen Systems läuft letztlich darauf hinaus, dass
es seine Einheitsversprechen nicht einlöse und als einende Totalität
untauglich sei. Sowohl der Nationalismus als auch der Menschheitsglaube
verkörpern ein Konzept der gesellschaftlichen Ganzheit, das Pluralismus und
Subjektautonomie negiert und Führerfiguren exponiert. Tollers Kritik gilt
lediglich der falschen Ausfüllung dieser Grundprinzips. Die Gemeinschaftsidee
wird bei ihm als Ideal aufgenommen und durch Abstraktion von
Gruppenunterschieden generalisiert. 

\HeadingThree{Die Totalität des Menschheitskultes}

Der expressionistische Menschheitskult im Stil der \Title{Wandlung} ist in
der Forschung ausgiebig kritisiert worden. Bemängelt wurden unter anderem das
hohle Pathos, die überzogenen sprachlichen Mittel und die Unglaubwürdigkeit
der Erweckungsszenen.\Footnote{\Abr{Vgl} \AbrPair{z}{B} \BibRef{sokel.81}{26}:
  \Cite{Tollers Optimismus [..] erscheint uns heute unglaublich kindlich.},
  \Page{28}: \Cite{Der Glaube an die Macht der Phrase ist naiv.},
  \BibRef{benson.87}{49}: \Cite{Tollers Idealismus grenzt hier an Naivität.}}
Dabei wurde leider selten darauf eingegangen, welche Funktion der
Verbrüderungsglaube im Sinne einer Bewältigung der gesellschaftlichen
Entwicklungen erfüllte und was ihn für eine ganze Generation von
expressionistischen Literaten (vorübergehend) zur respektablen Weltanschauung
werden ließ.

In der Konzeption des expressionistischen Menschheitskultes kommen eine Reihe
von Merkmalen zusammen, die in ihrer Summe ein äußerst attraktives
Weltdeutungsmodell für die zeitgenössischen Intellektuellen und Literaten
darstellen mussten. Vor dem Hintergrund der bereits dargestellten
Krisenerfahrungen moderner Subjektivität bietet es mit seinem metaphysischen
Verbindlichkeitsanspruch eine Art Religionsersatz. Die Gruppe der Individuen,
denen die mystischen Urkräfte als Verbrüderungspotential zugesprochen werden,
ist umfassend: Es geht um nicht weniger als \Quote{die Menschheit}. 
Die inneren Kräfte
sind unabhängig von eigener Tat, sie brauchen nur erweckt zu werden. Das
verschafft den bereits wissenden Intellektuellen, den \Quote{Geistigen}, ein
potentiell grenzenloses Publikum für eine universelle Botschaft, die sie als
Künstler individuell ausgestalten können. Mit dem Glauben an die
Menschheit wird von allen sozialen Ausdifferenzierungen der bestehenden
Gesellschaftsstruktur abstrahiert, so dass sich die Künstler \Cite{als
  Menschen} geradezu im Mittelpunkt des Volkes wähnen können. Eine Kunst, die
ohnehin nur noch gesellschaftliches Teilsystem ist und noch dazu im Krieg
praktisch nicht stattfindet, muss in solcher Universalität wohl eine Berufung
erkennen.  Die Wirkung der historischen Erfahrung des ersten Weltkrieges ist
als ausschlaggebender Auslöser der Bewegung zu sehen. Die Schlacht nationaler
Gruppen, die sich voneinander ideologisch abgrenzen und dabei allesamt
\Cite{als Menschen} zu Schaden kommen, hatte Europa umgewühlt und die
Akzeptanz der alten Gemeinschaftsbegriffe untergraben. Eva Kolinsky fasst dies
folgendermaßen zusammen:

\begin{BlockQuote}
  [Der] negative Bezug zur Gegenwart kulminiert im Krieg, der als Aufgipfelung
  der industriellen und \Cite{materiellen} Prinzipien des 19. Jahrhunderts
  erfahren wurde und er steigert sich zur Ablehnung der sozialen Wirklichkeit
  überhaupt.\Footnote{\BibRef{kolins.70}{59}.}
\end{BlockQuote}
Eben diese Ablehnung der sozialen Wirklichkeit konnte im Geiste durch die
totale Abstraktion der Menschheitsverbrüderung geleistet werden. Im Gegensatz
zu allen Gemeinschaftsbegriffen, die sich der sozialen Realität entnehmen
lassen, ermöglicht die Formel von der Menschheit eine subsumtive
Identitätsbildung ohne tatsächliche Unterordnungen zu vollziehen. Das liefert
dem \Cite{Künstler als Protagonist innerer
  Erneuerung}\Footnote{\Abr{Ebd}, \Page{68}.} eine universelle
Überschrift für seine Aktivitäten, die höchste Ansprüche mit
Gestaltungsfreiheit verbindet. Der Menschheitskult verspricht so einen idellen
Ausweg aus subjektiver und künstlerischer Isolation unter der flatternden
Fahne der Weltverbesserung.

Die Idee des \Cite{Menschheitskultes}, die kraft ihrer universellen
Abstraktion inhaltlich kaum Angriffspunkte bietet, ist also funktional für die
Situation, in der sie aufgebracht und rezipiert wurde. Sie bildet das Substrat
einer \Cite{gegenmodern} zu nennenden Geisteshaltung, die an der ideellen
Erschaffung einender, ganzheitlicher Konzeptionen für das Harmonieren einer
Welt interessiert ist, deren Komplexität sie zurücknehmen
möchte. Sie liefert einen Fokus für die Übertragung metaphysischer Restposten
aus den Sphären diskreditierter Religion und bedient intellektuelle
Behauptungswünsche in der Massengesellschaft.  

Die Universalität der \Quote{Menschheitsidee} und ihre kultisch-religiöse
Propagierung sind in Tollers \Title{Wandlung} paradigmatisch ausgestaltet. 
Das Stück ist ein Musterbeispiel für den Versuch, die als Verlust empfundene
Auflösung einer alle einenden intersubjektiven (göttlichen) Ordnungsinstanz
durch das neue Evangelium eines \Quote{Künstler-Priesters} zu ersetzen. Das
Potential umfassender Einheit konnte ein solches Evangelium nur erreichen,
indem es auf einem entsprechend universellen Gemeinschaftsbegriff gründete. 
\Cite{Die Menschheit} als denkbar umfassendste Kategorie ist in diesem Sinne
ein naheliegendes Konzept.

Eine solche totale Gemeinschaft, deren Gegebenheit man ideell immer schon
voraussetzt, \emph{praktisch herbeiführen} zu wollen, führt zu dem Widerspruch,
dass zwischen Führerfiguren und zu Erweckenden unterschieden werden und ein
belehrender Interaktionsprozess initiert werden muss. Dessen Inhalt ist aus
der schlichten Tatsache, dass da \Cite{Menschen} miteinander in Kontakt
treten, gar nicht zu erschließen. Dieses Dilemma ist in der
\Title{Wandlung} noch durch einen Erweckungs-Automatismus überdeckt, dessen
Schilderung dort endet, wo konkretes Handeln beginnt. Als Grundlage einer
sozialen Revolution taugt er eben nur abstrakt rhetorisch und nur
dann, wenn sich die zu erweckende Masse in ihre Rolle fügt.  
\emph{Konsensbildung} kann in diesem Gemeinschaftsmodell gar nicht
herbeigeführt werden, weil gemeinschaftliche Übereinstimmung immer schon
die Prämisse ist. Toller Bearbeitung dieses Grundwiderspruchs 
sollte denn auch für ein weiteres Drama reichen: \Title{Masse Mensch}.









