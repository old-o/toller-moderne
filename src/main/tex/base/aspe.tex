% Aspekte und Themenfelder der Drameninterpretation

\HeadingOne{Aspekte und Themenfelder der Dramenanalyse}{Aspekte und Themenfelder}

\HeadingTwo{Die ausgewählten Dramen im Kontext ihrer Zeit}

In Ernst Tollers Jugendzeit addierte sich die -- für die Jahrhundertwende
kennzeichnende -- \Cite{Identitätskrise des modernen
  Subjekts}\Footnote{\BibRef{thome.00}{25}. Der Beitrag versucht eine
  Charakterisierung des spezifischen Bewusstseinswandels in Deutschland um die
  Jahrhundertwende. Dabei kommt auch der systemtheoretische Ansatz, wie er in
  Kapitel 1 kurz vorgestellt wurde, zur Anwendung.} 
mit den besonderen Gegebenheiten seiner \Cite{sozialen Randlage}.\Footnote{Siehe
  \BibRef{rothe.83}{19f.}.} 
Als Sohn jüdischer Kaufleute in einem preußischen Provinzort in Posen gehörte
er zu einer Minderheit und er blieb auch während der Schul- und Studienzeit ein
Außenseiter. Tollers euphorische Meldung als Kriegsfreiwilliger gründete sich unter
anderem auf den Wunsch \Quote{dazu zu gehören}. Seine Rastlosigkeit
gegen Ende des Krieges und die Konfrontation mit den vielschichtigen 
urbanen Lebenswelten von Berlin und München taten ein Übriges, um die Frage
der \emph{sinngebenden sozialen Zugehörigkeit} zum Grundproblem seiner
Selbstreflexion werden zu lassen.

Tollers selbsgewählte Kriegsbeteiligung zeitigte Erfahrungen mit einer
Todesmaschinerie, deren sinnspendende Perspektivierung er nicht
lange aufrecht erhalten konnte. Die Desillusionierung wirkte politisierend. Sein
gefühlsmäßiger Pazifismus näherte sich bald dem Konzept eines durch \Cite{Erweckung}
zu bewirkenden Sozialismus. Die eigene Zuordnungs- und Weltdeutungsproblematik
verknüpfte er -- wie viele seiner Zeitgenossen -- mit den Verelendungsphänomenen
des Industriekapitalismus und machte sich das Ideal des \emph{\Cite{neuen
    Menschen}} zu eigen.

Konträr zu dem Phänomen der Moderne, dass \Cite{Selbstvergewisserung nur über
  einen prinzipiell nicht verallgemeinerungsfähigen Reflexionsprozess erreicht
  werden kann},\Footnote{\BibRef{thome.00}{22}.} beriefen sich die \Cite{'O
  Mensch'-Expressionisten} auf ein \Quote{Urwesen}, eine allgemeinste Form des
Menschseins, die sie als verbindende Kraft (re-)kultivieren wollten. Toller
gestaltete im Sinne dieser Grundhaltung seine Dramen \Title{Die Wandlung} und
\Title{Masse Mensch}. Das Ideal einer in Einheit und Ganzheit
verschmelzenden Menschengemeinschaft ist darin ergänzt durch die -- der
Philosophie Gustav Landauers entnommene -- \emph{Rollenverteilung zwischen den
  \Cite{Geistigen} und dem \Cite{Volk}}: Einzelne, sensible Intellektuelle,
die den \Cite{Geist} der Gemeinschaft aus früheren Zeiten in sich bewahrt
haben, läuten als verkündende, beispielgebende Leitfiguren eine
Erweckungsbewegung ein, die Landauer \Quote{Sozialismus}
nannte.\Footnote{Vgl. Gustav Landauers \Cite{Aufruf zum Sozialismus} von
  1911. Hier wiedergegeben gemäß \cite{ruehle.73}.}  Toller gestaltete diesen
Erweckungssozialismus als das am Ende der \Title{Wandlung} erreichte
revolutionäre Credo. Neben der subjektiven Sinnsuche wird dort auch eine
beißende Kritik des Krieges ausgestaltet: Die kriegsmäßige
\emph{Funktionalisierung von Medizin und Wissenschaft} kommt ebenso
ins Visier wie die \emph{inhumane Instrumentalisierung christlicher und
  traditioneller Werte}.

Ernst Toller griff als einer der Führer der Münchener Räterepublik aktiv in
die zeitgeschichtlichen Ereignisse ein, die den Übergang vom Weltkrieg zur
ersten deutschen Republik begleiteten. Nach der Niederschlagung der Revolution
ging er für fünf Jahre in Haft.  

Aus Tollers Vorstellungen von der Rollenverteilung zwischen geistigen Führern und
lenkbarer Volksmasse hatte sich im Zuge seiner Revolutionsbeteiligung ein
Problem ergeben: Die \Cite{Masse} war den falschen Führern gefolgt. Die
konkurrierende Sozialismuskonzeption, die statt Erweckung Klassenkampf
propagierte, wurde deswegen in \Title{Masse Mensch} als dramatischer
Gegenspieler der Verbrüderungsvision abgebildet.
Die innerrevolutionären Gegensätze ließen sich nur schwer in das Konzept der
Menschengemeinschaft eingliedern. Toller hielt zwar an dem idealistischen Kriterium
der Gewaltlosigkeit fest, doch die Zweifel an der Machbarkeit des Ideals saßen
tief.

Die Realitäten der Weimarer Republik, mit denen sich Toller nach seiner
Haftentlassung konfrontiert sah, ließen die expressionistischen Utopien noch
weiter in den Hintergrund treten. Die Problemlage seines spürbaren
\Quote{Modernisierungsrückstandes} arbeitete er in \Title{Hoppla, wir
  leben!} selbstkritisch auf. Die \emph{Auflösung der Einheitsperspektive},
die persönliche Verunsicherung durch rasante wirtschaftliche und technische
Veränderungen und die leidhafte Erfahrung von Oberflächlichkeit und Vereinzelung
bildeten den Stoff für dieses Drama. Es ist interessant, wie Toller aus diesen
Verlusterfahrungen eine neue politische Handlungsperspektive entwickeln konnte.

Seine relativ konstruktive Haltung gegenüber dem republikanischen System
änderte sich angesichts der erstarkenden NS-Bewegung zum warnenden Appell. Im
späteren Exil war ihm an der Bewahrung eines \Quote{aufrechten Gangs der
Anständigen} gelegen, den er in Form einer breiten antifaschistischen
Frontbildung auszuweiten
versuchte. In seinem letzten Drama stand demzufolge die \emph{Suche nach
Allianzmodellen des Widerstandes} im Vordergrund, die auf der persönlichen
Ebene als \Cite{Weg der Wahrheit} dargestellt und an einem vorbildhaften
Protagonisten exemplifiziert wurde.

\HeadingTwo{Das Grundmotiv der Einheit und Ganzheit}

In den begrifflichen Grundlegungen des vorhergehenden Kapitels wurde
herausgearbeitet, dass die \Cite{Dissoziation des Subjekts} als
mentalitätsgeschichtlicher Brennpunkt von intellektueller
Modernewahrnehmung im zeitlichen Umfeld von Spätexpressionismus und
Weltkriegsvorstimmung aufgefasst werden kann. Die Konfrontation mit dem
bloß Partikularen und Temporären sozialer Realität ist bei Ernst Toller durch seine
soziale \Cite{Randlage} noch verstärkt.\Footnote{Diese Einschätzung wird
  insbesondere durch die entsprechende Darstellung von Wolfgang Rothe nahe
  gelegt. \Abr{Vgl} \BibRef{rothe.83}{19}.}

Im dramatischen Schaffen Tollers ist das Streben nach \emph{Einheit und
Ganzheit} des Individuums \Cite{als Mensch} in der liebevollen Gemeinschaft
\Cite{der Menschheit} ein Grundmotiv, das unmittelbar aus den genannten
Dispositionen des modernen Subjekts heraus verstanden werden kann.
Dieses Grundmotiv wird der erste Ansatzpunkt der folgenden Dramenanalysen
sein. Anhand der geradezu prototypischen Durchgestaltung dieses
Gemeinschaftsmotivs im Drama \Title{Die Wandlung} werden genauere Facetten
und Bestimmungen dieser impliziten Modernereflexion Tollers
herausgearbeitet.

Das Spezifische am Gemeinschaftsideal in Tollers (früher) Dramatik ist die
konsequente \emph{Generalisierung} des Konzeptes von Einheit und Ganzheit: Innerhalb
einer umfassend begriffenen \Cite{Menschheit}, also der Gesamtheit aller Menschen,
werden gesellschaftliche Teilbereiche oder Leitideen vorgeführt und als unvollkommene
Gemeinschaftsträger qualifiziert. Sowohl durch ihre Begrenztheit und den
damit nötigen Ausschluss von Nicht-Zugehörigen, als auch durch innere
Ungerechtigkeit und Scheinheiligkeit bleiben diese Gemeinschaften \Cite{dem
Menschen} unangemessen.  

In den untersuchten Dramen finden sich hierfür ein ganze Reihe von Beispielen.
Anhand der Herkunftswelt in der \Title{Wandlung} wird gezeigt, wie eine
religiöse Teilgemeinschaft wie das Judentum als begrenzter gesellschaftlicher
Bezirk einengend wirkt und unfrei macht. Ein Teilaspekt ist dabei, dass die
kommerzzentrierte Wertegemeinschaft bürgerlicher Kreise auf sozialer Exklusion
und Konventionen basiert, die eher Heuchelei und Ignoranz als wahre
Gemeinschaft fördern. 

Als Scheinideal diskreditiert wird auch die
Nationalgemeinschaft des Vaterlandes, die nach außen \Cite{Fremde} ausgrenzt
und im Innern Ausbeutungsverhältnisse, Macht- und Besitzunterschiede
aufweist. Im Krieg tritt die Destruktivität der nationalen Gemeinschaft fatal
zu Tage, wie die Fronterlebnisse in der \Title{Wandlung} illustrieren.  

In \Title{Masse Mensch} wird dann dargestellt, dass auch die Gemeinschaft
proletarischer Revolutionäre potentiell von Massentrieben und falschen Führern
beeinträchtigt ist und ein gewalttätiges Verständnis des Klassengegensatzes
den gemeinschaftsstiftenden Sinn der Menschheitserneuerung untergraben kann.

Im nachrevolutionären Frieden von \Title{Hoppla, wir leben!} erweist sich das
Staatssystem schließlich als ambivalente Klammer für Klassengegensätze, die von den
Individuen nur durch tatkräftiges Engagement überwunden werden könnten. 

In der \Title{Wandlung} und in \Title{Masse Mensch} finden sich jeweils Protagonisten, die
diese Mängel erfahren, die Generalisierung des Gemeinschaftskonzepts
vollziehen und die Vision von umfassender Menschengemeinschaft verkünden.
Beide Hauptfiguren propagieren \Cite{die Liebe zu den Menschen}
als wahre Bestimmung jedes menschlichen Individuums. Das kompromisslose
Verständnis des Menschheitsbegriffs bewirkt ein Gemeinschaftsverständnis, das
keine Feinde haben kann und haben darf. Der kategorische Imperativ der \Cite{Liebe}
verbannt außerdem jeden inneren Unfrieden aus dieser Vision.

Es wird also untersucht, welche Faktoren für Tollers Gemeinschaftsideal konstitutiv
sind und in welches Verhältnis er das (intellektuelle) Individuum und seine
Gemeinschaft setzt. Die in der \Title{Wandlung} vollzogene Kontrastierung verschiedener
Formen von Gemeinschaft -- Familie, Religion, Liebesbeziehung, Nation und 
Menschheit -- ist ein wichtiger Gegenstand der Dramenanalyse.

Begleitend wird sich zeigen, wie das Grundmotiv die inhaltlichen
Themenfelder und die dramatische Gestaltung beeinflusst: Das für \Title{Die
Wandlung} bestimmende Verständnis von sozialer Revolution als
\Cite{Erweckung} und Selbstfindung oder die Postulierung von \Cite{Liebe} als
Fluidum menschlicher Gemeinschaft unter Abwertung ihrer persönlichen
(geschlechtlichen) Form sind dafür ebenso Beispiel wie die
Verwendung mystischer Handlungselemente aus Traum und Religion.

Im Anschluss an die Analyse der \Title{Wandlung} wird versucht, die als
implizite Modernereflexion Tollers herausgearbeiteten Konzepte und
Ideologeme, hinsichtlich ihrer \Cite{Modernität} zu charakterisieren. Dabei
wird der bereits in den vorhergehenden Kapitel dargestellte normative
Begriff \Cite{modernen Bewusstseins} zu Grunde gelegt. 

\HeadingTwo{Modifikation und Relativierung der Einheitsidee}

Die weiteren Dramen werden einerseits als Fortführung, Modifikation
oder Aufhebung der an der \Title{Wandlung} erarbeiteten Grundmotive
analysiert und damit in einen Entwicklungsbogen eingeordnet.
Andererseits wird die den Stücken jeweils spezifische Form von
Modernereflexion untersucht, um die Interpretation nicht durch
einseitige Orientierung am \Cite{messianischen Expressionismus}
einzuengen.

Für die Untersuchung von \Title{Masse Mensch} wird zunächst das
Motiv der \Quote{Einheit und Ganzheit} als hermeneutischer Ansatzpunkt
beibehalten. Das Stück wird im Wesentlichen als Problematisierung
der revolutionären Gemeinschaft betrachtet, in der das Ideal der
Menschheitsgemeinschaft mit dem der Klassengemeinschaft konkurriert.
Die Frage des Gewalteinsatzes und das Verhältnis von Revolutionären und
Staatsvertretern erscheinen als Konsequenzen dieser Problematik.

Bei \Title{Hoppla, wir leben!} ist eine Pluralisierung der
Figurenperspektiven und der Wegfall des positiv besetzten
messianischen Moments unübersehbar. Es wird deshalb untersucht, wie die
Gegenüberstellung von Pragmatismus und Aktionismus und die Kontrastierung von
Normalität und Irrsinn an die Stelle der messianischen Gemeinschaftsverkündung
treten. Die Konfrontation des Protagonisten mit einer Gesellschaft, die
einen mehrjährigen Entwicklungsvorsprung aufweist, ist dabei sozusagen
als dramatisierte Form schockartiger Modernisierungserfahrung von
besonderem Interesse.

Im letzten behandelten Stück \Title{Pastor Hall} steht die gesellschaftliche
Marginalisierung des ethischen Subjekts und seines Anspruchs auf
Freiheit und Anstand im Vordergrund. Da die Nazi-Herrschaft, die den
alles dominierenden Handlungsrahmen dieses Dramas darstellt, als
\Cite{Verwerfung} oder partielle Gegenläufigkeit des gesellschaftlichen
Modernisierungsprozesses einzuordnen ist, erhält die Analyse von
dramatischer Modernereflexion hier eine zusätzliche Problemebene.  


