% Examensarbeit: Literaturverzeicnis

\begin{BibList}{}

\bibitem[Altenhofer, 1981]{altenh.81}
  Rosemarie Altenhofer: 
  \Cite{Masse Mensch}.
  In: \BibRef{HERMAN.81}{129-141}.

\bibitem[Anz/Stark, 1994H]{ANZ.94}
  Thomas Anz, Michael Stark [Hrsg.]:
  Die Modernität des Expressionismus. 
  Stuttgart 1994.

\bibitem[Anz, 1994]{anz.94} 
  Thomas Anz: 
  Gesellschaftliche Modernisierung, literarische Moderne und philosophische
  Postmoderne. Fünf Thesen. 
  In: \BibRef{ANZ.94}{1-8}.

\bibitem[Bab, 1926]{bab.26}
  Julius Bab:
  Die Chronik des deutschen Dramas. Fünfter Teil. Deutschlands dramatische
  Produktion 1919-1926.
  Berlin 1926.

\bibitem[Bebendorf, 1990]{bebend.90}
  Klaus Bebendorf:
  Toller expressionistische Revolution. 
  Frankfurt/Main 1990.

\bibitem[Benson, 1987]{benson.87}
  Renate Benson:
  Deutsches expressionistisches Theater: Ernst Toller und Georg Kaiser.
  New York [u.a.] 1987. (Kanadische Studien zur deutschen
  Sprache und Literatur; Bd. 38).

\bibitem[Bütow, 1975]{buetow.75}
  Thomas Bütow:
  Der Konflikt zwischen Revolution und Pazifismus im Werk Ernst Tollers.
  Mit einem dokumentarischen Anhang: Essayistische Werke Tollers.
  Briefe von und über Toller. 
  Hamburg 1975.

\bibitem[Chen, 1998]{chen.98}
  Huimin Chen:
  Inversion of revolutionary ideals : a study of the tragic
  essence of Georg Büchner's \Cite{Dantons Tod}, Ernst
  Toller's \Cite{Masse Mensch}, and Bertolt Brecht's 
  \Cite{Die Massnahme}.
  New York 1998. (Studies on themes and motifs in
  literature ; 33)

\bibitem[Choluj, 1991]{choluj.91}
  Bozena Choluj:
  Deutsche Schriftsteller im Banne der Novemberrevolution 1918:
  Bernhard Kellermann, Lion Feuchtwanger, Ernst Toller, Erich
  Mühsam, Franz Jung.
  Wiesbaden 1991.

\bibitem[Chong, 1998]{chong.98}
  Dong-Lan Chong:
  Der literarisch-politische Diskurswandel in den Dramen Ernst
  Tollers. Die Konfigurations- und Symbolikanalyse der Dramen Ernst Tollers
  von 1917 bis 1927.
  Marburg 1998.

\bibitem[Denkler, 1981]{denkle.81}
  Horst Denkler:
  \Cite{Die Wandlung}. 
  In: \cite{HERMAN.81}\Page{116-128}.

\bibitem[Distl, 1993]{distl.93}
  Dieter Distl:
  Ernst Toller. Eine politische Biographie.
  Schrobenhausen 1993. (zugl. Dissertation,
  Universität München)

\bibitem[Dove, 1993]{dove.93}
  Richard Dove:
  Ernst Toller. Ein Leben in Deutschland. 
  Göttingen 1993.

\bibitem[Dove/Lamb, 1992]{DOVE.92}
  Richard Dove /
  Steven Lamb [Hrsg.]:
  German writers and politics : 1918 - 39. 
  Basingstoke 1992.
  (Warwick studies in the European humanities)

\bibitem[Droop, 1922]{droop.22}
  Fritz Droop:
  Ernst Toller und seine Bühnenwerke. Eine Einführung.
  Mit selbstbiographischen Notizen des Bühnendichters. 
  Berlin 1922.

\bibitem[Esselborn, 1994]{esselb.94}
  Hans Esselborn:
  Der literarische Expressionismus als Schritt zur Moderne.
  In: \cite{PIECHO.94}, Band 1, \Page{416-429}.

\bibitem[Fritton, 1986]{fritto.86}
  Michael Hugh Fritton:
  Literatur und Politik in der Novemberrevolution 1918/1919.
  Theorie und Praxis revolutionärer Schriftsteller in Stuttgart
  und München (Edwin Hörnle, Fritz Rück, Max Barthel, Ernst Toller, Erich
  Mühsam).
  Frankfurt/Main 1986.

\bibitem[Fuld/Ostermaier, 1996]{FULD.96}
  Werner Fuld /
  Albert Ostermaier [Hrsg.]:
  Die Göttin und ihr Sozialist: Christiane Grautoffs Autobiographie - ihr
  Leben mit Ernst Toller.
  Bonn 1996.

\bibitem[Grimminger, 1995]{grimin.95}
  Rolf Grimminger:
  Aufstand der Dinge und der Schreibweisen. Über Literatur und Kultur der
  Moderne.
  In: \BibRef{GRIMIN.95}{12-41}.

\bibitem[Grimminger et al., 1995H]{GRIMIN.95}
  Rolf Grimminger /
  Jurij Murasov /
  Jörn Stückrath [Hrsg.]:
  Literarische Moderne. Europäische Literatur im 19. und 20. Jahrhundert.
  Reinbek 1995.

\bibitem[Grunow-Erdmann, 1994]{grunow.94}
  Cordula Grunow-Erdmann:
  Die Dramen Ernst Tollers im Kontext ihrer Zeit. 
  Heidelberg 1994.

\bibitem[Habermas, 1981]{haberm.81}
  Jürgen Habermas:
  Die Moderne -- ein unvollendetes Projekt.
  In: Habermas, Jürgen: Kleine politische Schriften. Frankfurt/Main 1981,
  \Page{444-464}.

\bibitem[Habermas, 1985]{haberm.85}
  Jürgen Habermas:
  Der philosophische Diskurs der Moderne. Zwölf Vorlesungen. 
  Frankfurt/Main 1985.

\bibitem[Hermand, 1981]{herman.81}
  Jost Hermand:
  \Cite{Hoppla, wir leben!}.
  In: \BibRef{HERMAN.81}{161-178}.

\bibitem[Hermand, 1981H]{HERMAN.81}
  Jost Hermand [Hrsg.]:
  Zu Ernst Toller. Drama und Engagement. 
  Stuttgart 1981.

\bibitem[Hoffmann, 1968H]{HOFFMA.68}
  Ludwig Hoffmann [Hrsg.]:
  Erwin Piscator: Das Politische Theater. Faksimiledruck der Erstausgabe 1929.
  Berlin 1968.

\bibitem[Hohendahl, 1967]{hohend.67}
  Peter Uwe Hohendahl:
  Das Bild der bürgerlichen Welt im expressionistischen Drama.
  Heidelberg 1967.

\bibitem[Jauß, 1989]{jauss.89}
  Hans Robert Jauß:
  Studien zum Epochenwandel der ästhetischen Moderne.
  Frankfurt/Main 1989.

\bibitem[Kändler, 1981]{kaendl.81}
  Klaus Kändler:
  Zwischen Masse und Mensch -- Ernst Toller von der \Cite{Wandlung} bis
  \Cite{Hoppla, wir leben!} und \Cite{Feuer aus den Kesseln!}.
  In: \BibRef{HERMAN.81}{87-115}.

\bibitem[Kane, 1987]{kane.87}
  Martin Kane:
  Weimar Germany and the limits of political art. A study of the
  work of George Grosz and Ernst Toller. 
  Tayport 1987.

\bibitem[Kemper, 1998]{kemper.98}
  Dirk Kemper:
  Ästhetische Moderne als Makroepoche.
  In: \BibRef{VIETTA.98}{97-126}.

\bibitem[Kim, 1998]{kim.98}
  Hye Suk Kim:
  Der Wandel der Wertsysteme in Ernst Tollers Dramen.
  Passau 1998. (Dissertation, Universität Passau)

\bibitem[Klein, 1968]{klein.68}
  Dorothea Klein:
  Der Wandel der dramatischen Darstellungsform im Werk Ernst Tollers (1919 -
  1930).
  Bochum 1968. (Dissertation, Ruhr-Universität Bochum)

\bibitem[Kolinsky, 1970]{kolins.70}
  Eva Kolinsky:
  Engagierter Expressionismus. Politik und Literatur zwischen Weltkrieg und
  Weimarer Republik.
  Stuttgart 1970.

\bibitem[Koopmann, 1997]{koopma.97}
  Helmut Koopmann:
  Deutsche Literaturtheorien zwischen 1889 und 1920.
  Darmstadt 1997.

\bibitem[Kuhn, 1992]{kuhn.92}
  Tom Kuhn:
  Forms of Conviction: The Problem of Belief in Anti-Fascist Plays by
  Bruckner, Toller and Wolf
  In: \BibRef{DOVE.92}{163-177}.

\bibitem[Lamb, 1986]{lamb.86}
  Stephen Lamb:
  Hero or Villain? Notes on the Reception of Ernst Toller in the GDR. 
  In: \Emph{The German Quarterly}. Vol. 59. No. 3. 
  Cherry Hill 1986. \Page{375-386}.

\bibitem[Leydecker, 1998]{leydec.98}
  Karl Leydecker: 
  The laughter of Karl Thomas: Madness and politics in the
  first version of Ernst Toller's \Cite{Hoppla, wir leben!}.
  In: \Emph{Modern Language Review}. Vol. 93 (1). 
  Leeds 1998. \Page{121-132}.

\bibitem[Lohmeier, 2000a]{lohmei.00.1}
  Anke-Marie Lohmeier:
  Literatur im Modernisierungsprozess: Möglichkeiten einer
  modernisierungsgeschichtlichen Periodisierung. Vortrag, IVG-Kongress. 
  Wien 2000.

\bibitem[Lohmeier, 2000b]{lohmei.00.2}
  Anke-Marie Lohmeier:
  Vom unendlichen Ende des Volksmagisters. Die Intellektuellen, die
  \Cite{Massen} und die offene Gesellschaft.
  IASL Diskussionsforum online: 
  http://iasl.uni-muenchen.de/discuss/lisforen/lohmeier.htm

\bibitem[Luhmann, 1980]{luhman.80}
  Niklas Luhmann:
  Gesellschaftsstruktur und Semantik. Studien zur Wissenssoziologie der
  modernen Gesellschaft. Band 1.
  Frankfurt/Main 1980.

\bibitem[Neuhaus et al., 1999H]{NEUHAU.99}
  Stefan Neuhaus / 
  Rolf Selbmann / 
  Thorsten Unger [Hrsg.]:
  Ernst Toller und die Weimarer Republik. Ein Autor im Spannungsfeld von
  Literatur und Politik.
  Würzburg 1999. (Schriften der Ernst-Toller-Gesellschaft ; 1)

\bibitem[Ossar, 1998]{ossar.98}
  Michael Ossar:
  Anarchism in the dramas of Ernst Toller. The realm of necessity
  and the realm of freedom.
  Albany/NY 1980.
  
\bibitem[Piechotta et al., 1994H]{PIECHO.94}
  Hans Joachim Piechotta / 
  Ralph-Rainer Wuthenow /
  Sabine Rothemann [Hrsg.]:
  Die literarische Moderne in Europa. 3 Bände.
  Opladen 1994.

\bibitem[Piscator, 1929]{piscat.29}
  Erwin Piscator:
  Die Begegnung mit der Zeit. \Cite{Hoppla, wir leben!}. 3. September bis
  7. November 1927.
  In: \BibRef{HOFFMA.68}{146-160}.

\bibitem[Plumpe, 1993]{plumpe.93}
  Gerhard Plumpe:
  Ästhetische Kommunikation der Moderne. Band 2: Von Nietzsche bis zur Gegenwart.
  Opladen 1993.

\bibitem[Plumpe, 1995]{plumpe.95}
  Gerhard Plumpe:
  Epochen moderner Literatur. Ein systemtheoretischer Entwurf.
  Opladen 1995.

\bibitem[Reimers, 2000]{reimer.00}
  Kirsten Reimers:
  Das Bewältigen des Wirklichen. Untersuchungen zum dramatischen
  Schaffen Ernst Tollers zwischen den Weltkriegen. 
  Würzburg 2000.

\bibitem[Röttger, 1996]{roettg.96}
  Evelyn Röttger:
  Schriftstellerisches und politisches Selbstverständnis in Ernst
  Tollers Exildramatik.
  In: \Emph{Zeitschrift für deutsche Philologie}. Bd. 115. 
  Berlin 1996. \Page{239-261}.

\bibitem[Rothe, 1983]{rothe.83}
  Wolfgang Rothe:
  Ernst Toller. In Selbstzeugnissen und Bilddokumenten.
  Reinbek 1983.

\bibitem[Rothstein, 1987]{rothst.87}
  Sigurd Rothstein:
  Der Traum von der Gemeinschaft. Kontinuität und Innovation in Ernst Tollers
  Dramen. 
  Frankfurt/Main 1987.

\bibitem[Rühle, 1973]{ruehle.73}
  Günther Rühle:
  Zeit und Theater. Vom Kaiserreich zur Republik. 1913-1925.
  Berlin 1973.

\bibitem[Rühle, 1974]{ruehle.74}
  Günther Rühle:
  Zeit und Theater. Diktatur und Exil. 1933-1945.
  Berlin 1974.

\bibitem[Schneider, 1995]{schnei.95}
  Falko Schneider:
  Filmpalast, Varieté, Dichterzirkel. Massenkultur und literarische Elite in
  der Weimarer Republik.
  In: \BibRef{GRIMIN.95}{453-478}.

\bibitem[Schreiber, 1997]{schrei.97}
  Birgit Schreiber:
  Politische Retheologisierung. Ernst Toller frühe Dramatik als Suche nach
  einer \Cite{Politik der reinen Mittel}. 
  Würzburg 1997.

\bibitem[Sokel, 1981]{sokel.81}
  Walter Sokel:
  Ernst Toller.
  In: \BibRef{HERMAN.81}{25-40}. 

\bibitem[Spalek/Willard, 1989]{spalek.89}
  John M. Spalek / Penelope D. Willard: 
  Ernst Toller.
  In: Deutschsprachige Exilliteratur seit 1933. 
  Hrsg. von John M. Spalek und Joseph Strelka. 
  Bern 1989. \Page{1723-1765}.

\bibitem[Thomé, 2000]{thome.00}
  Horst Thomé: 
  Modernität und Bewusstseinswandel in der Zeit des Naturalismus und des Fin
  de siècle. 
  In: Mix, York-Gothart: Naturalismus, Fin de siécle, Expressionismus --
  1890-1918. München 2000. \Page{15-27}.
  (Hanser Sozialgeschichte der deutschen Literatur ; Band 7). 

\bibitem[Unger, 1996]{unger.96}
  Thorsten Unger:
  Antifaschistischer Widerstand und kulturelle Erinnerung im exilpolitischen
  Drama : Zu Ernst Tollers \Cite{Pastor Hall}.
  In: \Emph{Jahrbuch für Internationale Germanistik}. Reihe A. Bd. 40.
  Bern 1996. \Page{289-316}.

\bibitem[Vietta, 1992]{vietta.92}
  Silvio Vietta:
  Die literarische Moderne. Eine problemgeschichtliche Darstellung der
  deutschsprachigen Literatur von Hölderlin bis Thomas Bernhard.
  Stuttgart 1992.

\bibitem[Vietta, 1998]{vietta.98}
  Silvio Vietta:
  Die Modernekritik der ästhetischen Moderne.
  In: \BibRef{VIETTA.98}{97-126}.

\bibitem[Vietta/Kemper, 1994]{vietta.94}
  Silvio Vietta /
  Dirk Kemper:
  Expressionismus. 5. Auflage.
  München 1994.
  (UTB 362 ; Deutsche Literatur im 20. Jahrhundert, Band 3)

\bibitem[Vietta/Kemper, 1998H]{VIETTA.98}
  Silvio Vietta / 
  Dirk Kemper [Hrsg.] :
  Ästhetische Moderne in Europa. Grundzüge und Problemzusammenhänge seit der
  Romantik. 
  München 1998.

\bibitem[Wehler, 1987]{wehler.87}
  Hans-Ulrich Wehler:
  Deutsche Gesellschaftgeschichte. Band 1. Vom Feudalismus des Alten
  Reiches bis zur defensiven Modernisierung der Reformära. 1700-1815.
  München 1987.

\end{BibList}

%%% Local Variables: 
%%% mode: latex
%%% TeX-master: "MAIN"
%%% End: 
